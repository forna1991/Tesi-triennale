\chapter{Conclusioni}

    La tesi propone un metodo per identificare e successivamente calcolare i parametri delle curve catturate da una webcam posta in posizione frontale rispetto al robot.

    L'algoritmo di pathshape è in grado di identificare una linea all'interno dell'immagine e di ottenere un insieme di punti che stanno su di essa. Questo insieme di informazioni è stato usato come punto di partenza per il lavoro descritto. Abbiamo quindi iniziato osservando l'algoritmo stesso per aggiustare alcuni difetti scoperti durante le fasi iniziali. Successivamente il lavoro è proseguito con la costruzione di un sistema in grado di riconoscere la curvatura nella linea identificata nell'immagine. A questo punto dopo aver separato e isolato le varie curve presenti nella stessa, è stata implementata una funzione in grado di calcolarne i parametri. \'E infine stata aggiunta un'ulteriore funzione in grado di calcolare la distanza tra due curve aventi parametri differenti. I dati ottenuti nel passo precedente vengono quindi confrontati con i parametri delle curve presenti in memoria. Come ultimo passo viene restituito un set di curve simili a quella considerata.

    Per verificare la precisione con cui questi algoritmi operano è stata adottata una tecnica di testing basata su simulazioni utilizzando MATLAB. Sono stati quindi valutati la correttezza dei parametri calcolati dall'algoritmo di identificazione della curva e il tempo di esecuzione della parte di riconoscimento della stessa. Questa valutazione è stata svolta in relazione al tempo di esecuzione dell'algoritmo di pathshape.

    I risultati riguardanti il tempo di esecuzione sono stati senza dubbio ottimi. Il tempo di esecuzione della parte di codice relativa all'identificazione della curva e al pattern matching è risultato molto basso. Il rapporto tra il tempo necessario a eseguire la parte alla quale ho contribuito e quello necessario a eseguire l'algoritmo di pathshape è stato valutato in media inferiore a 1/100. Quindi per ogni unità di tempo impiegata per identificare e trovare una curva corrispondente, il pathshape impiega solitamente più di 100 unità di tempo per l'elaborazione dell'immagine e l'estrapolazione della linea.

    I risultati ottenuti sulla parte di identificazione dei parametri sono invece risultati non del tutto soddisfacenti. Infatti, è stato ottenuto un errore nel calcolo dei parametri della curva distribuito come segue:
    \begin{itemize}
        \item Nel 70\% dei casi riguardanti dati ricavati da curve lontane al massimo due metri dal robot è stato riscontrato un errore inferiore al 10\%.
        \item Nel 78\% dei casi riguardanti dati ricavati da curve a una distanza compresa tra i 2 e i 2.5 metri è stato riscontrato un errore inferiore al 20\%.
        \item Nel 21\% dei casi riguardanti distanze superiori ai 2.5 metri risulta invece molto difficile ottenere dati precisi e l'errore nel calcolo dei parametri della curva si riduce a valori inferiori al 20\%. Questo errore è dovuto ai limiti riscontrati nell'algoritmo di pathshape, i quali rendono quasi impossibile lavorare oltre distanze simili.
    \end{itemize}
    Un'ulteriore problema riscontrato nell'algoritmo utilizzato riguarda particolari tipi di curve aventi un angolo di curvatura inferiore a circa 20°, dove l'errore di calcolo si attesta intorno al 20\% solo nel 50\% dei test. Nonostante ciò, i risultati possono essere comunque sfruttati dopo una fase di pianificazione del percorso adeguata.

    Possiamo concludere affermando che l'algoritmo che esegue la parte di matching della curva identificata nell'immagine risulta efficace. Dai dati è risultata infatti una percentuale di riconoscimento piuttosto alta, nonostante gli errori apparsi nella fase preliminare.

\section{Lavori futuri}
    Uno dei primi aspetti che necessita una fase di miglioramento è l'affidabilità dell'algoritmo di pathshape. Questo può essere ottenuto aumentando sia la distanza a cui la linea viene identificata dall'algoritmo, sia la precisione su distanze superiori ai 2 metri. Ciò permetterebbe di eseguire l'algoritmo di identificazione della curva sulla base di dati più affidabili, ottenendo risultati migliori di quelli riscontrati attualmente. Infatti, l'aumento di precisione aiuterebbe ad identificare la curva con più sicurezza. In aggiunta l'aumento della distanza di identificazione permetterebbe lo studio dell'intera curva, consentendo un utilizzo più efficace dei parametri riguardanti l'angolo di curvatura.
    Una valida alternativa a questa soluzione è quella di migliorare la parte preliminare di correzione dei dati, in modo tale da poter lavorare con valori più regolari. Tuttavia, questa soluzione non risolve il problema della distanza di identificazione.

    Il potenziamento dell'algoritmo di pathshape, adottando tecniche di clustering quali \cite{linedetect}, risulterà molto più attuabile dopo la sostituzione della scheda embedded utilizzata con una versione più potente. Inoltre, l'utilizzo di una GPGPU\footnote{\url{http://gpgpu.org/}} (General Purpose GPU) permetterebbe di paralelizzare meglio le fasi di elaborazione delle immagini, migliorando ulteriormente le prestazioni dell'algoritmo. Infine un'altra ottimizzazione consiste nel migliorare l'algoritmo che processa e marca i punti dell'algoritmo di pathshape. L'obiettivo è quello di gestire un maggior numero di errori riguardanti i valori restituiti dall'algoritmo di pathshape, aumentandone quindi l'efficacia e l'affidabilità.
