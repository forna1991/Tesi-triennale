\chapter{L'identificazione della linea}

	Esistono molte applicazioni nell'ambito della robotica che richiedono al robot la capacità di muoversi in modo autonomo. Questo obiettivo risulta notevolmente semplificato con l'aggiunta di un marcatore nell'ambiente, in grado di suggerire un possibile percorso o delimitare una corsia. \'E quindi possibile far muovere il robot facendo si che mantenga una distanza fissa dal marcatore, permettendogli di seguire il percorso.
	Tuttavia se il percorso inizia a incurvarsi, mantenere il robot a distanza fissa dal marcatore potrebbe non essere la scelta migliore.

	Per questo motivo è stato sviluppato un algoritmo, denominato \textit{pathshape} \cite{pathshape}. Questo algoritmo è in grado di identificare un marcatore, nello specifico una linea nera di spessore noto, analizzando immagini raccolte da una webcam. Questo permette al robot di ottenere informazioni utili sul percorso che ha di fronte a se.

\section{Analisi dell' algoritmo di pathshape}

	L' algoritmo di pathshape ha la necessità di elaborare le immagini provenienti da uno stream video e analizzarle in modo da identificare un possibile marcatore all'interno di esse. Essendo il codice eseguito su una scheda con processore non particolarmente potente (\textasciitilde1GhZ) e avendo a disposizione poca RAM(\textasciitilde1GB), le performance per questo algoritmo sono un aspetto fondamentale. Per questo motivo l'algoritmo è stato scritto in linguaggio C++ in virtù delle ottime prestazioni offerte.

\section{L'algoritmo di pathshape}

	L'algoritmo di pathshape si basa essenzialmente su 4 fasi principali.
	In una prima fase la parte di piano visibile nell'immagine catturata dalla telecamera frontale viene processata attraverso l'algoritmo di Canny\cite{canny}, in grado di rimuovere la maggior parte degli elementi visibili nell'immagine, mantenendo solamente i contorni.

	A questo punto la nuova immagine contenente solo i contorni degli oggetti viene proiettata su un piano orizzontale, in modo da rendere più efficente questa fase. \'E quindi definito un set di posizioni $V$ e viene creata una telecamera virtuale che inizia a scorrere l'immagine proiettata posizionandosi nelle posizioni definite nel set $V$. Su ognuna delle immagini generate per la telecamera virtuale utilizza l'algoritmo RANSAC\cite{ransac} (RANdom SAmple and Consensus) per trovare due linee parallele a distanza nota tra loro (rapprentanti i margini della linea utilizzata come marcatore). L'algoritmo presi due punti casuali all'interno del set ne calcola la retta passante per entrambi. Quindi determina il sottoinsieme $S1$ come il numero dei punti passanti vicino alla retta appena calcolata entro un certo errore. Se il numero di punti trovati supera una certa soglia $|S1|>n$ l'algoritmo crea un modello adeguato usando i punti dell'insieme $S1$. Questo processo viene ripetuto più volte per aumentare le chance di ottenere un risultato corretto essendo RANSAC un algoritmo non deterministico.

	Dopo che due soluzioni soddisfacenti sono state trovate le soluzioni vengono confrontate tra di solo per verificare che siano effettivamente parallele e alla distanza prevista. In caso positivo viene calcolato un punto tra le due soluzioni che verrà usato dall'algoritmo di pathshape come punto di partenza. In questo posizione di partenza idoneo, dove le due linee rappresentanti i contorni della linea a nera a terra vengono identificate correttamente e alla giusta distanza, l'algoritmo comincia a scorrere la linea trovata effettuando piccoli passi di entità fissa \footnote{il valore di $pshapeStep$ che definisce l'entità di questi passi è stato fissato a 0.04-0.06m} in direzione della linea. Nel caso in cui la linea abbia un andamento curvilineo risulta più complesso in quanto in questo caso l'algoritmo RANSAC deve comunque adattare una retta alla serie di punti. Essendo questi non perfettamente in linea potrebbero non rientrare nei limiti di tolleranza imposti. Questo rende necessario avvicinare la telecamera virtuale al terreno cosi che le linee della curva appaiano più diritte, semplificando il processo di adattamento.

	Raggiunta la fine della parte di linea visibile nell'immagine, o dopo comunque una certa distanza l'algoritmo RANSAC non riuscirà più a trovare correttamente le linee. A questo punto l'algoritmo di pathshape riporta la posizione della telecamera frontale al punto di partenza e inizia a svolgere lo stesso procedimento in direzione opposta. Questo per completare il set di punti che andranno poi a identificare la linea.
